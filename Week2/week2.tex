\documentclass[a4paper, 11pt]{article}

\usepackage{amsmath}
\usepackage{amsfonts}
\usepackage{amsthm}
\usepackage{mathrsfs}
\usepackage{tikz}
\usepackage{pgfplots}
\usepackage{algorithm2e}
\usepackage{multicol}


\newtheorem{theorem}{Theorem}
\newtheorem*{corollary}{Corollary}
\newtheorem*{remark}{Remark}

\begin{document}
\section*{3.2 Kraft and McMillan Inequalities}
    \indent{} \textbf{length} is simply the number of characters in a string given an alphabet ($\mathcal{A}$).
    given a set $\mathcal{S}$ with the encoding: $f:\mathcal{S}\rightarrow \mathcal{A}^*$. The set of word lengths then denoted as:
    \[\left\{length(f(\mathcal{S})) : s \in \mathcal{S}\right\}\]

    \vspace{3mm}

\begin{theorem}[Kraft's Inequality]
    Let the set $\mathcal{S}$ of source words have $m$ elements. Then, let the encoding alphabet $\mathcal{A}$ have $n$ characters.
    A condition that there exists an \textbf{instantaneous uniquely decipherable} code $f:\mathcal{S} \rightarrow \mathcal{A}^*$ with lengths
    $\ell_1, \ell_2, \dots, \ell_3$ is:
    \begin{align}
        \sum_{i=1}^{m} \frac{1}{n^{\ell_i}} \leq 1
    \end{align}
\end{theorem}

\begin{theorem}[Mcmillan's Inequality]
    Let the set $\mathcal{S}$ of source words have $m$ elements. Then, let the encoding alphabet $\mathcal{A}$ have $n$ characters.
    A condition that there exists a \textbf{uniquely decipherable} code $f:\mathcal{S} \rightarrow \mathcal{A}^*$ with lengths
    $\ell_1, \ell_2, \dots, \ell_3$ is:
    \begin{align}
        \sum_{i=1}^{m} \frac{1}{n^{\ell_i}} \leq 1
    \end{align}
\end{theorem}

\begin{corollary}
    If there is a uniquely decipherable code with given word lengths, then there is an \emph{instantaneous} (uniquely decipherable) code with those word lengths.
\end{corollary}

\begin{remark}
    The Inequalities give absolute limits on the size of the encoding words necessary to encode a 'vocabulary' $\mathcal{S}$ of source words with a certain size. These are independent of any probablistic considerations.
\end{remark}

\indent{} There is a proof involved in this, but I have omitted it because I don't want to deal with it. That being said, since \emph{any} uniquely decipherable code must satisfy the Inequality, an \emph{instantaneous} one should as well.

\section*{3.3 Noiseless Coding Theorem}

\end{document}